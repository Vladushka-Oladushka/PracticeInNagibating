\documentclass[a4paper,12pt]{article} 
\usepackage{amsfonts,amssymb} 
\usepackage[utf8]{inputenc} 
\usepackage[english, russian]{babel} 
\usepackage[top=2cm, 
left=3cm, 
right=1cm, 
bottom=2cm]{geometry}

\usepackage{amsthm} 
\usepackage{amsmath} 
\usepackage[pdftex]{graphicx} 
\usepackage{epstopdf} 
\usepackage{float} 
\usepackage{subfigure} 
\usepackage[nottoc,numbib]{tocbibind} 
\usepackage{graphicx} 
\newcommand\tab[1][1cm]{\hspace*{#1}}
\begin{document}
\begin{center}
\begin{LARGE}
\textbf{Сложность}\\
\end{LARGE}
\end{center}

\footnotesize{ 
\hspace*{7.2 cm}
%
Врач, строитель и программистка спорили о том,\\\hspace*{7.94 cm}чья профессия древнее. Врач заметил: "В Библии \\\hspace*{7.94 cm}сказано, что Бог сотворил Еву из ребра Адама.\\\hspace*{7.94 cm}Такая операция может быть проведена только\\\hspace*{7.94 cm}хирургом, поэтому я по праву могу утверждать,\\\hspace*{7.94 cm}что моя профессия самая древняя в мире". Тут\\\hspace*{7.94 cm}вмешался строитель и сказал: "Но еще раньше в\\\hspace*{7.94 cm}Книге Бытия сказано, что Бог сотворил из хаоса\\\hspace*{7.94 cm}небо и землю. Это было первое и, несомненно,\\\hspace*{7.94 cm}наиболее выдающееся строительство. Поэтому,\\\hspace*{7.94 cm}дорогой доктор, вы не правы. Моя профессия\\\hspace*{7.94 cm}самая древняя в мире". Программистка при этих\\\hspace*{7.94 cm}словах откинулась в кресле и с улыбкой\\\hspace*{7.94 cm}произнесла: "А кто же по-вашему сотворил\\\hspace*{7.94 cm}хаос?" \\
}
\section{Сложность, присущая программному обеспечению }
\subsection{Простые и сложные программные системы}
\begin{small}

\parindent=1.25cm
\hspace{1.25 cm}
Звезда в преддверии коллапса; ребенок, который учится читать; клетки крови, атакующие вирус, - это только некоторые из потрясающе сложных объектов физического мира. Компьютерные программы тоже бывают сложными, однако их сложность совершенно другого рода. Брукс пишет: "Эйнштейн утверждал, что должны существовать простые объяснения природных процессов, так как Бог не действует из каприза или по произволу. У программиста нет такого утешения: сложность, с которой он должен справиться, лежит в самой природе системы" [1]. \par 
\medskip
Мы знаем, что не все программные системы сложны. Существует множество программ, которые задумываются, разрабатываются, сопровождаются и используются одним и тем же человеком. Обычно это начинающий программист или профессионал, работающий изолированно. Мы не хотим сказать, что все такие системы плохо сделаны или, тем более, усомниться в квалификации их создателей. Но такие системы, как правило, имеют очень ограниченную область применения и короткое время жизни. Обычно их лучше заменить новыми, чем пытаться повторно использовать, переделывать или расширять. Разработка подобных программ скорее утомительна, чем сложна, так что изучение этого процесса нас не интересует. \par \medskip
Нас интересует разработка того, что мы будем называть промышленными программными продуктами. Они применяются для решения самых разных задач, таких, например, как системы с обратной связью, которые управляют или сами управляются событиями физического мира и для которых ресурсы времени и памяти ограничены; задачи поддержания целостности информации объемом в сотни тысяч записей при параллельном доступе к ней с обновлениями и запросами; системы управления и контроля за реальными процессами (например, диспетчеризация воздушного или железнодорожного транспорта). Системы подобного типа обычно имеют большое время жизни, и большое количество пользователей оказывается в зависимости от их нормального функционирования. В мире промышленных программ мы также встречаем среды разработки, которые упрощают создание приложений в конкретных областях, и программы, которые имитируют определенные стороны человеческого интеллекта. \par \medskip
Существенная черта промышленной программы - уровень сложности: один разработчик практически не в состоянии охватить все аспекты такой системы. Грубо говоря, сложность промышленных программ превышает возможности человеческого интеллекта. Увы, но сложность, о которой мы говорим, по-видимому, присуща всем большим программных системам. Говоря "присуща", мы имеем в виду, что эта сложность здесь неизбежна: с ней можно справиться, но избавиться от нее нельзя. \par \medskip
Конечно, среди нас всегда есть гении, которые в одиночку могут выполнить работу группы обычных людей-разработчиков и добиться в своей области успеха, сравнимого с достижениями Франка Ллойда Райта или Леонардо да Винчи. Такие люди нам нужны как архитекторы, которые изобретают новые идиомы, механизмы и основные идеи, используемые затем при разработке других систем. Однако, как замечает Петерс: "В мире очень мало гениев, и не надо думать, будто в среде программистов их доля выше средней" [2]. Несмотря на то, что все мы чуточку гениальны, в промышленном программировании нельзя постоянно полагаться на божественное вдохновение, которое обязательно поможет нам. Поэтому мы должны рассмотреть более надежные способы конструирования сложных систем. Для лучшего понимания того, чем мы собираемся управлять, сначала ответим на вопрос: почему сложность присуща всем большим программным системам?
\end{small}
\subsection{Почему программному обеспечению присуща сложность?}
\begin{small}
\parindent=1.25cm
\hspace{1.25 cm}Как говорит Брукс, "сложность программного обеспечения - отнюдь не случайное его свойство" [3]. Сложность вызывается четырьмя основными причинами: \par
$\bullet$ \tab сложностью реальной предметной области, из которой исходит заказ на разработку;\par
•	\tab трудностью управления процессом разработки;\par
•	\tab необходимостью обеспечить достаточную гибкость программы;\par
•	\tab неудовлетворительными способами описания поведения больших дискретных систем.\par
\textbf{Сложность реального мира.} Проблемы, которые мы пытаемся решить с помощью программного обеспечения, часто неизбежно содержат сложные элементы, а к соответствующим программам предъявляется множество различных, порой взаимоисключающих требований. Рассмотрим необходимые характеристики электронной системы многомоторного самолета, сотовой телефонной коммутаторной системы и робота. Достаточно трудно понять, даже в общих чертах, как работает каждая такая система. Теперь прибавьте к этому дополнительные требования (часто не формулируемые явно), такие как удобство, производительность, стоимость, выживаемость и надежность! Сложность задачи и порождает ту сложность программного продукта, о которой пишет Брукс.\par \medskip
Эта внешняя сложность обычно возникает из-за "нестыковки" между пользователями системы и ее разработчиками: пользователи с трудом могут объяснить в форме, понятной разработчикам, что на самом деле нужно сделать. Бывают случаи, когда пользователь лишь смутно представляет, что ему нужно от будущей программной системы. Это в основном происходит не из-за ошибок с той или иной стороны; просто каждая из групп специализируется в своей области, и ей недостает знаний партнера. У пользователей и разработчиков разные взгляды на сущность проблемы, и они делают различные выводы о возможных путях ее решения. На самом деле, даже если пользователь точно знает, что ему нужно, мы с трудом можем однозначно зафиксировать все его требования. Обычно они отражены на многих страницах текста, "разбавленных" немногими рисунками. Такие документы трудно поддаются пониманию, они открыты для различных интерпретаций и часто содержат элементы, относящиеся скорее к дизайну, чем к необходимым требованиям разработки.\par \medskip
Дополнительные сложности возникают в результате изменений требований к программной системе уже в процессе разработки. В основном требования корректируются из-за того, что само осуществление программного проекта часто изменяет проблему. Рассмотрение первых результатов - схем, прототипов, - и использование системы после того, как она разработана и установлена, заставляют пользователей лучше понять и отчетливей сформулировать то, что им действительно нужно. В то же время этот процесс повышает квалификацию разработчиков в предметной области и позволяет им задавать более осмысленные вопросы, которые проясняют темные места в проектируемой системе.\par \medskip
Большая программная система - это крупное капиталовложение, и мы не можем позволить себе выкидывать сделанное при каждом изменении внешних требований. Тем не менее даже большие системы имеют тенденцию к эволюции в процессе их использования: следовательно, встает задача о том, что часто неправильно называют \textit{сопровождением программного обеспечения}. Чтобы быть более точными, введем несколько терминов:\par
•	\tab под \textit{сопровождением} понимается устранение ошибок;\par
•	\tab под \textit{эволюцией} - внесение изменений в систему в ответ на изменившиеся требования к ней;\par
•	\tab под \textit{сохранением} - использование всех возможных и невозможных способов для поддержания жизни в дряхлой и распадающейся на части системе.\par \medskip
К сожалению, опыт показывает, что существенный процент затрат на разработку программных систем тратится именно на сохранение. \par \medskip
\textbf{Трудности управления процессом разработки.} Основная задача разработчиков состоит в создании иллюзии простоты, в защите пользователей от сложности описываемого предмета или процесса. Размер исходных текстов программной системы отнюдь не входит в число ее главных достоинств, поэтому мы стараемся делать исходные тексты более компактными, изобретая хитроумные и мощные методы, а также используя среды разработки уже существующих проектов и программ. Однако новые требования для каждой новой системы неизбежны, а они приводят к необходимости либо создавать много программ "с нуля", либо пытаться по-новому использовать существующие. Всего 20 лет назад программы объемом в несколько тысяч строк на ассемблере выходили за пределы наших возможностей. Сегодня обычными стали программные системы, размер которых исчисляется десятками тысяч или даже миллионами строк на языках высокого уровня. Ни один человек никогда не сможет полностью понять такую систему. Даже если мы правильно разложим ее на составные части, мы все равно получим сотни, а иногда и тысячи отдельных модулей. Поэтому такой объем работ потребует привлечения команды разработчиков, в идеале как можно меньшей по численности. Но какой бы она ни была, всегда будут возникать значительные трудности, связанные с организацией коллективной разработки. Чем больше разработчиков, тем сложнее связи между ними и тем сложнее координация, особенно если участники работ географически удалены друг от друга, что типично в случае очень больших проектов. Таким образом, при коллективном выполнении проекта главной задачей руководства является поддержание единства и целостности разработки. \par \medskip


\end{small}
\end{document}