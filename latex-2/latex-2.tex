\documentclass[a4paper,12pt]{article} 
\usepackage{amsfonts,amssymb} 
\usepackage[utf8]{inputenc} 
\usepackage[english, russian]{babel} 
\usepackage[top=2cm, 
left=3cm, 
right=1cm, 
bottom=2cm]{geometry}

\usepackage{amsthm} 
\usepackage{amsmath} 
\usepackage[pdftex]{graphicx} 
\usepackage{epstopdf} 
\usepackage{float} 
\usepackage{subfigure} 
\usepackage[nottoc,numbib]{tocbibind} 
\usepackage{graphicx} 
\newcommand\tab[1][1cm]{\hspace*{#1}}
\begin{document}

\begin{LARGE}
\textbf{12}\\
\end{LARGE}


Для визначення тилу регресії необхідно скористатися тестом Г.ЧОУ. Для цього введемо деякі позначення і сформуемо з них таблицю. Ціею таблицею можна користуватися для випадку трендів, від лінійного.\\
Модель з декилькох рівнянь:
\begin{center}
\begin{tabular}{cccccc}
№ & вид & кількість & залишкова & к-сть пар-трів & к-сть ступенів свободи \\
рівняння: & рів-ня & спостережень & сума квадратів & у рівнянні & залишкової дисперсії \\
(1) & $y^1$ & $n_1$ & $S^1$ & $p_1$ & $n_1-p_1$\\
(2) & $y^2$ & $n_2$ & $S^2$ & $p_2$ & $n_2-p_2$\\
\end{tabular}
\end{center}
Рівняння тренду за всіею сукупністю даних:
\begin{center}
\begin{tabular}{cccccc}
(3) \tab & $y^3$ \tab \tab & $n$ \tab \tab & $S^3$ \tab \tab & $p_3$\tab  \tab & $n-p_2$ \tab \\
\end{tabular}
\end{center}
Якщо розглядаеться модель лінійна, то $p_1=p_2=p_3=2$.\\
Залишкова сума квадратів на кусково-лінійной моделі дорівнює:$S=S^1+S^2$, а відповідна кількість ступенів свободи є:\\
$(n_1-p_1)+(n_2-p_2)=(n-p_1-p_2)$\\
При переході від одного рівняння тренду до кусково-лінійної моделі скорочення залишкової суми квадратів  таке: $\Delta S=S^3-S$.\\
Кількість ступенів свободі, що відповідає $\Delta S$ дорівнює:
$n-p_3-(n-p_1-p_2)=p_1+p_2-p_3$\\
Розрахуеємо $F_{CT}$:
$F_{CT}=\dfrac{\Delta S \, (n-p_1-p_2)}{S \, (p_1+p_2-p_3)}$\\
Якщо $F_{CT}>F_{KP}=F(\alpha;p_1+p_2-p_3;n-p_1-p_2)$, то з йомовірностю $(1-\alpha)*100\%$ гіпотеза про структурну стабільність тенденції відхіляеться, а вплив структурних змін на динаміку ряду зважають значущим. А це означає, що необхідно будувати кусково-лінійну модель.\\
Якщо $F_{CT}<F_{KP}$, то модель структурно стабільна, тобто рівняння $(1)$ і $(2)$ описують одну і ту тенденцію і тому відмінність коефіцієнтів цих рівнянь статистично незначима.\\
Далі розглянемо випадкі структурної нестабільності тенденції. Для лінійної тенденції  $y^1=a_1+b_1t $, а $ y^2=a_2+b_2t $.\\
1. Якщо $a_1$ і $a_2$ значимо відрізняються, а $b_1$ i $b_2$ статистично значимо різняться, це геометрічно означає, що прямі $y^1$ і $y^2$ паралельні. Тобто відбулася стрибкоподібна зміна рівнів ряду $y_t$ в момент часу $t^*$. При незміному середньому абсолютному прирості\\
2. Параметрі $a_1$ і $a_2$ статистично незначимо відрізняються, а $b_1$ і $b_2$ статистично значимо відрізняються. Це означає, що $y^1$ і $y^2$ перетинають вісь ординат в одній точці, а зміна тенденції пов'язана зі зміною середнього абсолютного приросту з моменту $t^*$\\
3. $a_1$ і $a_2$, $b_1$ і $b_2$ значимо відрізняються. У цьому випадку початкові рівні і середні абсолютні прирости різні.\\
Розглянемо метод Д. Гуйараті дослідження тенденції для вище розглянутого випадку. Д. Гуйараті включає до моделі регресії  фіктову змінну $z_t=\begin{cases}1,\: t<t^*,\\
0,\: t>t^*.\\
\end{cases}
$
Далі розглянемо модель:\\
$y_t=a+bz_t+ct+dz_tt+\varepsilon_t$(*)\\
Получим: $y_t=\begin{cases}(a+b)+(c+d)t+\varepsilon_t,\: t<t^*,\\
a+ct+\varepsilon_t,\: t>t^*.\\
\end{cases}
$\\
Якщо порівняти рівняння $y^1$ i $y^2$ з (*), то отримаємо, що $a_1 =(a+b),\: b_1=c+d; a_2=a_1, b_2=c.$\\
Отже, $b=a_1-a_2,\: d=b_1-b_2$. Тоді оцінка статистичної значущості параметрів $b$ i $d$ рівняння (*).\\
Маємо, якщо в рівнянні (*) $b$ є статистично значущим, а $d$-ні, то випадок 2, якщо $b$ i $d$ статистично значущі, то отримаємо випадок 3.\\
Подхід Д. Гуйараті простіший, так як використовується для дослідження тільки одне рівняння.\\
\section{Перевірка відповідності розподілу спостережень нормального розподілу}
1. Критерій Девіда-Хартл-Пірсона(RS-критерій)
Критерій нормальності розподілу ймовірності випадкової величини грунтується на розподілу відношення розкіду до стандартного відхилення. Нехай розглядаеється вибірка $x_1, x_2,...x_N$.\\
Статистика критерія має вигляд: $U_{CT}=\dfrac{R}{S}$, де $R=x_{max}-x_{min},\: S$ - стандартне відхилення вибірки. Гіпотеза нормальності приймається, якщо $U_1(\alpha)<U_{CT}<U_2(\alpha), \alpha$ - рівень значущості.\\
2. Критерій значущості Фроціні\\
Фроціні запропунував простий, але достатно потужний критерій нормальності з параметрами, що оцінюються за вибіркою, і грунтуєтся на статистиці\\
$B_N=\dfrac{1}{\sqrt{N}}\sum\limits^N_{i=1} |\Phi(z_i)-\dfrac{i-0,5}{N}|$, де $z_i=\dfrac{x_i-\overline{x}}{S}$, $\overline{x}=\frac{1}{N}\sum\limits^N_{i=1} x_i$, $S^2=\frac{1}{N-1}\sum\limits^N_{i=1}(x_i-\overline{x})^2$, $\Phi(z_i)$ - функція розподілу $N(0,1)$.\\
Якщо $B_N<B_N(\alpha)$, то гіпотеза про нормальність розподілу випадкових величин не відхиляється. Критичні значення $B_N$ наведенi в таблиці
\end{document}